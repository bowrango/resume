\documentclass[11pt,a4paper,sans]{moderncv}
\moderncvstyle{banking}
\moderncvcolor{black}

\usepackage[scale=0.915]{geometry}
\usepackage{enumitem}
\usepackage[T1]{fontenc}
\usepackage{lmodern}

\setlist[itemize]{
  label=\textbullet,
  leftmargin=0.45cm,
  itemsep=0cm,  
  parsep=0cm,
  topsep=0cm,
}

\newcommand{\role}[4]{
  \vspace{-0.2cm}
    \begin{tabular*}{1\textwidth}[t]{l@{\extracolsep{\fill}}r}
      {\bfseries#3} & #2 \\ #1 & #4
    \end{tabular*}
    \vspace{-0.2cm}
}

\newcommand{\project}[2]{
  \vspace{0cm}
    \begin{tabular*}{1\textwidth}[t]{l@{\extracolsep{\fill}}r}
      \textbf{#1} & #2
    \end{tabular*}
    \vspace{-0.5cm}
}

\newcommand{\tools}[1]{
    \par #1
}

\name{Matt}{Bowring}

\begin{document}

\makecvtitle

\vspace{-1.5cm}
\begin{center}
    \href{mailto:mbowring@purdue.edu}{\texttt{mbowring@purdue.edu}} $|$ Cambridge, MA \\
    {\href{https://www.mattbowring.com}{\color{blue} Website}} $|$
    {\href{https://github.com/bowrango}{\color{blue} Github}} $|$
    \href{https://www.linkedin.com/in/matt-bowring-463610149/}{\color{blue} LinkedIn} 
\end{center}
\vspace{-1cm}

\section{Experience}

\role{Software Engineer (Math/PDE)}{May 2022 - Present}{The MathWorks}{Natick, MA (on-site)}
\tools{MATLAB, C++, Python, CMake, OpenMP, Git}

{\begin{itemize}
\item Lead development of the {{\color{blue}\href{https://www.mathworks.com/help/matlab/quantum-computing.html?s_tid=CRUX_lftnav}{MATLAB Quantum Computing Library}}} from an unreleased prototype to 9,100+ users

\begin{itemize}[label=-]
\item Design object-oriented MATLAB framework that enables quantum programming on Windows, Linux, and Mac
\item Develop templated C++ state-vector simulator for CPU/GPU with multi-threading using OpenMP
\item Implement graph compiler/transpiler using C++ to generate optimized assembly code of quantum circuits
\item Build client-side REST infrastructure using AWS and IBM cloud services to manage async jobs on quantum hardware
\item Develop sparse algorithms to compute expectation values of observables and decompose quantum gates
\item Manage CI/CD, write all unit/functional/performance tests, and benchmark against other quantum libraries 
\end{itemize}

\item Consult industry users on machine learning and quantum computing applications

\begin{itemize}[label=-]
\item Implemented quantum neural networks for image classification ({{\color{blue}\href{https://www.spiedigitallibrary.org/conference-proceedings-of-spie/13451/1345105/Optimizing-supervised-quantum-machine-learning-for-pixel-classification/10.1117/12.3052526.short}{SPIE publication}}}) and reinforcement learning
\item Researched constraint encodings using quadratization and penalty dephasing to benchmark optimization heuristics
\item Developed presentations and examples for industry conferences (ACS 25, AWS re: Invent 25, IBM QDC 22-25)
\end{itemize}

\end{itemize}}

\vspace{0.2cm}

\role{Software Engineer}{May 2021 - May 2022}{The MathWorks}{Natick, MA (on-site)}
\tools{MATLAB, C++, Python, Git}

{\begin{itemize}
\item Trained recurrent graph network on QM7-X molecular dataset to predict equilibrium energy and classify optimal configurations; Developed format-agnostic data pipeline to encode atomic properties using OpenBabel
\item Developed REST interface to cloud-based quantum annealing hardware; Benchmarked minor-embedding heuristics on constrained combinatorial problems; Wrote parallel simulations to compute qubit transition probabilities for truncated Hamiltonian based on graph-theoretic formalism across coupling strength and energy separation parameters 
\end{itemize}}

\vspace{-0.5cm}

\section{Education}

\role{M.S. Mechanical Engineering}{Jan. 2025 - Present}{Purdue University (3.8 GPA)}{West Lafayette, IN (remote)}

{\begin{itemize}
\item Fabricated mixed-signal PCB to solve combinatorial problems using phase dynamics of LC oscillators (\href{https://arxiv.org/abs/2512.23720}{\color{blue} arXiv preprint}); Accepted for research mentorship under Stephen Wolfram

\begin{itemize}[label=-]
\item Developed simulations of coupled oscillators in LTspice and modeled noisy phase dynamics with MATLAB
\item Designed schematics and layouts for two multi-layer boards in Altium Designer and soldered components
\item Developed real-time embedded C++ firmware using SPI and FreeRTOS to measure phases at 1MHz
\item Automated test instrument control to inject perturbations and analyze Fourier spectrum using MATLAB 
\end{itemize}
\end{itemize}}

\vspace{0.2cm}

\role{B.S. Mechanical Engineering}{Aug. 2017 - May 2021}{The University of New Hampshire (3.9 GPA)}{Durham, NH (on-site)}

{\begin{itemize}
\item Awarded over \$100,000 in scholarships for developing an autonomous quadcopter
\item Led the Quadcopter Engineering Team; Served as Makerspace Administrator, Academic Tutor and Mentor
\end{itemize}}

\vspace{-0.5cm}

\section{Projects}

\project{Computer Systems}{2024 - Present}
{\begin{itemize}
\item Built custom multi-GPU Linux machine for PyTorch/CUDA research experiments; Configured Beelink home server using NixOS and Tailscale
\end{itemize}} 

\project{Machine Learning}{2020 - Present}
{\begin{itemize}
\item Wrote packet parser to process and analyze financial data using Python and Wireshark; Trained recurrent convolutional network to predict dash-camera misalignment using OpenCV and PyTorch; Built data pipeline and Bayesian network to predict trends for online video game using NetworkX and PyTorch
\end{itemize}} 

\project{Robotics}{2019 - 2021}
{\begin{itemize}
\item Integrated PX4 flight-stack with Raspberry Pi to enable waypoint tracking for quadcopters; Implemented ROS/MAVROS communication, telemetry, and interfaced Gazebo for SITL simulation with Python; Tuned flight controller and analyzed motor responses using MATLAB; 3D-printed frame, soldered electrical components, and conducted field tests to evaluate controller performance
\end{itemize}} 

\end{document}