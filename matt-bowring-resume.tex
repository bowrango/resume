\documentclass[11pt,a4paper,sans]{moderncv}
\moderncvstyle{banking}
\moderncvcolor{black}

\usepackage[scale=0.915]{geometry}
\usepackage{enumitem}

\setlist[itemize]{
  label=\textbullet,
  leftmargin=0.5cm,
  itemsep=0cm,  
  parsep=0cm,
  topsep=0cm,
}

\newcommand{\role}[4]{
  \vspace{-0.1cm}
    \begin{tabular*}{1\textwidth}[t]{l@{\extracolsep{\fill}}r}
      {\bfseries#3} & #2 \\ #1 & #4
    \end{tabular*}
    \vspace{-0.2cm}
}

\newcommand{\project}[2]{
  \vspace{0cm}
    \begin{tabular*}{1\textwidth}[t]{l@{\extracolsep{\fill}}r}
      \textbf{#1} & #2
    \end{tabular*}
    \vspace{-0.5cm}
}

\newcommand{\tools}[1]{
    \par #1
}

\name{Matt}{Bowring}

\begin{document}

\makecvtitle

\vspace{-1.5cm}
\begin{center}
    \href{mailto:mbowring@purdue.edu}{\texttt{mbowring@purdue.edu}} $|$ Cambridge, MA \\
    {\href{https://www.mattbowring.com}{\color{blue}  Website}} $|$
    {\href{https://github.com/bowrango}{\color{blue} Github}} $|$
    \href{https://www.linkedin.com/in/matt-bowring-463610149/}{\color{blue} LinkedIn} 
\end{center}
\vspace{-1cm}

\section{Experience}

\role{Software Engineer (Math/PDE)}{May 2022 - Present}{The MathWorks}{Natick, MA (on-site)}
\tools{MATLAB, C++, Python, CMake, OpenMP, CUDA, Git, Linux/Windows OS}

{\begin{itemize}
\item Lead development of the {{\color{blue}\href{https://www.mathworks.com/help/matlab/quantum-computing.html?s_tid=CRUX_lftnav}{MATLAB Quantum Computing Library}}}; Design the core object-oriented framework that enables quantum programming natively in MATLAB; Develop numerical algorithms to simulate/compile/optimize gate-based quantum circuits, generate assembly code, and compute expected values of observables; Build client-side REST infrastructure using AWS and IBM cloud services to manage async jobs on quantum hardware; Write all unit/functional/performance tests, integrate/benchmark third-party quantum software, and manage CI/CD; Grew the library from an unreleased prototype to 9,100+ users.
\item Consult industry users with machine learning and quantum computing applications; Developed a quantum neural network for image classification ({{\color{blue}\href{https://www.spiedigitallibrary.org/conference-proceedings-of-spie/13451/1345105/Optimizing-supervised-quantum-machine-learning-for-pixel-classification/10.1117/12.3052526.short}{SPIE publication}}}) and researched methods to encode hard/soft optimization constraints; Featured at industy conferences (IBM QDC 2022-2025, AWS re: Invent 2025).
\end{itemize}}

\role{Software Engineer}{May 2021 - May 2022}{The MathWorks}{Natick, MA (on-site)}
\tools{MATLAB, C++, Python, Git, Linux/Windows OS}

{\begin{itemize}
\item Trained a recurrent graph network on the QM7-X molecular dataset to predict equilibrium energy and classify optimal configurations; Developed a format-agnostic data pipeline to encode atomic properties using OpenBabel.
\item Developed a REST interface to cloud-based quantum annealing hardware; Evaluated binary optimization problems with hard/soft constraints to benchmark minor-embedding algorithms; Wrote parallel simulations to compute the energy spectrum of a graph-theoretic qubit coupled to external fields.
\end{itemize}}

\vspace{-0.5cm}

\section{Education}

\role{M.S. Mechanical Engineering}{Jan. 2025 - Present}{Purdue University (3.8 GPA)}{West Lafayette, IN (remote)}

{\begin{itemize}
% \item Research interest surround physics-based optimization (\href{https://gigabug.org/posts/bug8/}{\color{blue} preprint}).
\item Fabricated a mixed-signal PCB to solve combinatorial optimization problems using phase dynamics of coupled oscillators (\href{https://arxiv.org/abs/2512.23720}{\color{blue} arXiv preprint}); Developed Lyapunov stability models using LTspice and MATLAB; Designed schematics and layouts in Altium Designer; Wrote MATLAB interfaces to my test instruments for signal processing and verification; Developed real-time embedded C++ firmware using SPI and FreeRTOS; Accepted for research mentorship under Stephen Wolfram.
\item Judged the annual hackathon at MIT; Attended the Hot Chips conference at Stanford.
\end{itemize}}

\vspace{0.2cm}

\role{B.S. Mechanical Engineering}{Aug. 2017 - May 2021}{The University of New Hampshire (3.9 GPA)}{Durham, NH (on-site)}

{\begin{itemize}
\item Awarded over \$100,000 in scholarships for developing an autonomous quadcopter. \item Lead the Quadcopter Engineering Team;  
Selected for Makerspace Administrator, Academic Tutor and Mentor.
\end{itemize}}

\vspace{-0.5cm}

\section{Projects}

\project{Computer Systems}{2024 ‐ Present}
{\begin{itemize}
      \item Built a custom multi-GPU Linux machine for PyTorch/CUDA research experiments; Configured a Beelink home server using NixOS and Tailscale.
\end{itemize}} 

\project{Machine Learning}{2020 ‐ Present}
{\begin{itemize}
      \item Wrote a packet parser to process and analyze financial data using Python and Wireshark; Trained a recurrent network to predict dash-camera misalignment using OpenCV and PyTorch; Built a data pipeline and Bayesian network to predict trends for an online video game using NetworkX and PyTorch.
\end{itemize}} 

\project{Robotics}{2019 ‐ 2021}
{\begin{itemize}
      \item Integrated the PX4 flight-stack with a Raspberry Pi to enable waypoint tracking for quadcopters; Implemented ROS/MAVROS communication, telemetry, and interfaced Gazebo for SITL simulation with Python; Tuned the flight controller and analyzed motor responses using MATLAB; 3D-printed the frame, soldered electrical components, and conducted field tests.
\end{itemize}} 

\project{Mechanical Design}{2020 ‐ 2021}
{\begin{itemize}
      \item Lead a student team to develop an air intake for a mock jet turbine; Developed CAD models using SolidWorks, ran CFD simulations with Ansys, and 3D-printed airfoils; Experimented with aluminum casting using a custom silicon mold and vacuum chamber. \href{https://www.gla.ac.uk/myglasgow/news/peopleprojects/2017mar-apr/headline_522468_en.html}{\color{blue} Collaborated with the University of Glasgow}.
\end{itemize}} 

\end{document}